\chapter*{List of Symbols}										% da *-Variante m�ssen Kopfzeilen und TOC-Eintrag von Hand generiert werden
\markboth{List of Symbols}{List of Symbols} 				% Kopfzeile manuell anpassen
\addcontentsline{toc}{chapter}{List of Symbols}				% TOC-Eintrag



\section*{Abbreviations}
%
\acrodef{AV}[AV]{Atrioventricular}
\acrodef{BTD}[BTD]{Bridging to decision}
\acrodef{BTR}[BTR]{Bridging to recovery}
\acrodef{BTT}[BTT]{Bridging to transplantation}
\acrodef{BVAD}[BVAD]{Biventricular Assist Device}
\acrodef{CHR}[CHR]{Chien Hrones Reswick}
\acrodef{CO}[CO]{Cardiac Output}
\acrodef{CVDs}[CVDs]{Cardiovascular Diseases}
\acrodef{CVS}[CVS]{Cardiovascular System}
\acrodef{DT}[DT]{Destination therapy}
\acrodef{EDV}[EDV]{End-diastolic volume}
\acrodef{EF}[EF]{Ejection fraction}
\acrodef{ESV}[ESV]{End-systolic volume}
\acrodef{ILC}[ILC]{Iterative learning control}
\acrodef{IMACS}[IMACS]{International Mechanically Assisted Circulatory Support}
\acrodef{INTERMACS}[INTERMACS]{Interagency Registry for Mechanically Assisted Circulatory Support}
\acrodef{HiL}[HiL]{Hardware in the Loop}
\acrodef{HR}[HR]{Heart rate}
\acrodef{HTx}[HTx]{Heart transplantation}
\acrodef{LVAD}[LVAD]{Left Ventricular Assist Device}
\acrodef{MCL}[MCL]{Mock circulatory loop}
\acrodef{MCS}[MCS]{Mechanical Circulatory Support}
\acrodef{RWTH}[RWTH]{Rheinisch-Westf{\"a}lische Technische Hochschule}
\acrodef{RVAD}[RVAD]{Right Ventricular Assist Device}
\acrodef{SL}[SL]{Semilunar}
\acrodef{SV}[SV]{Stroke Volume}
\acrodef{VADs}[VADs]{Ventricular Assist Devices}
\acrodef{WHO}[WHO]{World Health Organization}
\acrodef{ZN}[ZN]{Ziegler Nichols}

%* \acs{Name} ruft explizit die Abk�rzung auf.
%* \acl{Name} ruft explizit den ausgeschriebenen Begriff auf.

\begin{tabularx}{\textwidth}{p{.18\textwidth}X}
\acs{AV} & \acl{AV} \\
\acs{BTD} & \acl{BTD} \\
\acs{BTR} & \acl{BTR} \\
\acs{BTT} & \acl{BTT} \\
\acs{BVAD} & \acl{BVAD} \\
\acs{CHR} & \acl{CHR} \\
\acs{CO} & \acl{CO} \\
\acs{CVDs} & \acl{CVDs} \\
\acs{CVS} & \acl{CVS} \\
\acs{DT} & \acl{DT} \\
\acs{EDV} & \acl{EDV} \\
\acs{EF} & \acl{EF} \\
\acs{ESV} & \acl{ESV} \\
\acs{HiL} & \acl{HiL} \\
\acs{HR} & \acl{HR} \\
\acs{HTx} & \acl{HTx} \\
\acs{ILC} & \acl{ILC} \\
\acs{IMACS} & \acl{IMACS}\\
\acs{INTERMACS} & \acl{INTERMACS}\\
\acs{LVAD} & \acl{LVAD} \\
\acs{MCL} & \acl{MCL} \\
\acs{MCS} & \acl{MCS} \\
\acs{RWTH} & \acl{RWTH}\\
\acs{SL} & \acl{SL}\\
\acs{SV} & \acl{SV} \\
\acs{VADs} & \acl{VADs} \\
\acs{WHO} & \acl{WHO} \\
\acs{ZN} & \acl{ZN} \\
\end{tabularx}
%
% \section*{Physikalische Gr��en}
%
% \begin{tabularx}{\textwidth}{p{.18\textwidth}Xp{.1\textwidth}}
% $\mathrm{v}$ & Geschwindigkeit & $\frac{km}{h}$ \\
% $\mathrm{t}$ & Zeit & $h$\\
% \end{tabularx}
%
% \section*{Mathematische Gr��en}
%
% \begin{tabularx}{\textwidth}{p{.18\textwidth}X}
% $\mathrm{M}$ & mathematische Beispielgr��e \\
% \end{tabularx}
%
% \section*{Indizes}
%
% \begin{tabularx}{\textwidth}{p{.18\textwidth}X}
% $k$ & Anzahl der Proze�schritte \\
% $P_{1}...P_{n}$ & Proze�schritt $P_{1}$ bis $P_{n}$\\
% $V_{Ziel}$ &  Zielvektor\\
% \end{tabularx}
%
% \section*{Konstanten}
%
% \begin{tabularx}{\textwidth}{p{.18\textwidth}X}
% $\pi$ & 3.141592653589 \\
% \end{tabularx}
