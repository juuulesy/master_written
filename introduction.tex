\chapter{Introduction}
\section{Motivation and goal}
Heart failure is considered one of the most common cardiovascular diseases. The standard therapy for this disease is characterized by drug therapy and, last but not least, the implantation of a donor organ. However, since the number of patients who are dependent on a donor heart exceeds the number of available donor organs by far, which results in death of about $25\,\%$ of the patients waiting for a donor organ, it is necessary to develop new and improve already existing alternative forms of therapy \cite{VAD2}.
In case drug therapy is not sufficient to keep the patient alive until a heart transplantation can be performed, mechanical heart support systems of various kinds are increasingly used. Depending on the device used, the heart function can either be completely taken over by the device as would be the case with total artificial hearts, or a relief of the heart can be achieved by support via a ventricular assist device.
\\Within this work, the axial flow pump Sputnik VAD, which was developed in Russia, is used. This is an implantable left ventricular assist device providing continuous flow support used in bridging therapy until transplantation.
As until now studies are not clear on whether a continuous flow profile introduces physiological difficulties for the patient,
the aim of this work is to implement a flow control for the Sputnik VAD using Matlab and Simulink, which allows to adjust the flow to different predefined flow trajectories, continuous as well as pulsatile, by means of the continuous support system \cite{VAD3}. This goal will be achieved by implementing and evaluating different control algorithms. The testing of which will be performed using a cardiovascular system simulator developed at MedIT.
\section{Thesis structure}
This thesis aims to initially provide the reader with the necessary fundamental knowledge on the human cardiovascular system as well as basic information on the clinical syndrome of heart failure. Chapter 2 of this thesis, furthermore, provides an introduction the therapeutic objective and the technology of ventricular assist devices.
\\Following this, the basics of control engineering are explained in chapter 3. In particular, the structure and notation of a standard control loop, the operation and design of PI controllers, as well as the functionality and implementation of iterative learning controls are discussed.
\\Chapter 4 provides information on the Sputnik VAD and the circulatory system simulator of the Chair for Medical Information Technology. Furthermore, a system identification is performed through the acquisition of static maps for different test fluids and connection tube lengths.
\\The main part of this work is represented by chapter 5. This chapter deals with the implementation and evaluation of various control algorithms. At first two PI controllers are tuned, implemented and evaluated. Performance of these two controllers is compared in order to determine which controller provides the higher performance. The higher performing PI controller is then used as a stabilizing controller for an iterative learning control (ILC) implementation in form of a parallel architecture. Initially, various tests are performed to design the Q filter of the ILC. After successfully defining a cut off frequency for the filter, it is tested by using different reference flow trajectories. Initially, these tests are performed for repetitive disturbances, which are represented as a heartbeat of constant frequency using the test rig. Then, the ability of the ILC to also suppress non-repetitive disturbances in the form of variable heart rates is tested. These tests are of relevance for a later use of the system on the patient, since a constant heart rate cannot be guaranteed for the patient. In order to increase the performance of the ILC for this kind of disturbances, the standard ILC will finally be extended by a function which allows for resampling of the data to the iteration length corresponding to the current heart rate. For this implementation, tests are performed using different reference trajectories and the results are compared.
\\Finally, chapter 6 presents a summary of the findings of this thesis and offers an outlook on possible extensions of this thesis.
