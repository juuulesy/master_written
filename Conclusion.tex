\chapter{Conclusion and future work}
\section{Conclusion}
As the gap between the incidence of cardiovascular disease and the availability of donor organs will continue to grow in the coming years due to demographic change, it is of increasing importance to further develop alternative therapies for CVDs.
For this reason, a flow control system for the Sputnik VAD axial flow pump was developed in the context of this work, enabeling the system to follow different predefined flow trajectories.
\\This was achieved by a stepwise development and optimization of a control loop, which comprises a parallel architecture of a PI controller and an ILC.
\\First, the performance of two PI controllers, designed using the Ziegler Nichols and Chien Hrones Reswick methods, was compared. The second one showed a significantly higher performance, which led to its further use within the ILC implementation as a stabilizing feedback controller.
\\After successful optimization of the Q-filter implemented within the ILC, a series of tests were performed to determine the quality of the ILC. Under the idealized assumption of disturbance by a heartbeat with regular heart rate of $60\, bpm$ and adjustment of the length of the flow trajectories to a duration of $1\,s$ per iteration, satisfactory results were obtained with averaged error values ranging from $RMSE_{\mathrm{rect}}=0.22\,l/min$ for a rectangular trajectory to $RMSE_{\mathrm{sine}}=0.09\,l/min$ for a sinusoidal trajectory. When the standard ILC was subjected to a non-repetitive disturbance in the form of heartbeats of variable heart rates, a significant degradation in performance was evident. For this case, the averaged error values amounted to $RMSE_{\mathrm{rect}}=0.77\,l/min$ for the rectangular signal. For a constant reference flow, it peaked at $RMSE_{\mathrm{const}}=1.13\, l/min$.
\\In order to ensure high performance even in the presence of variable disturbances, the ILC was extended by further data preprocessing in the form of resampling of the ILC's actuating variable data to an iteration length corresponding to the current heart rate. This allowed the averaged error values for a constant flow to be reduced to just $RMSE_{\mathrm{resampled,const}}=0.05\, l/min$. A noticeable improvement in performance was also achieved for a sinusoidal and a triangular signal with RMSE reductions of $\Delta{RMSE}=0.6\,l/min$ and $\Delta{RMSE}=0.41\,l/min$, respectively. Only for the rectangular signal no clear increase in performance could be seen.
For all reference trajectories tested, however, the pump stopped after some time because the actuating variable exhibited oscillating behavior.

\section{Future work}
In order to ensure long-term flow control under the influence of non-repeating disturbances, as would be necessary for use on patients, the robustness of the controller to this type of disturbance would have to be optimized.

Conceivable approaches to achieve this goal could be given in the use of other ILC implementations, such as a current iteration learning control. Also, the use of a higher-order learning algorithm, which uses not only the last but also even earlier iterations for the calculation, would be recommendable.

Since the rotor of the pump stopped for all tested reference trajectories of the revised ILC, it would be advisable to use an alternative control unit in order to improve robustness of the control system. One possible hardware change would be to use the original control unit associated with the Sputnik VAD instead of the ESCON 50/4 EC-S.

For employment of the resampling functionality, as implemented in this work to improve the ILC for non-repetitive disorders, the additional analysis of ECG data to gain information on the heart rate would be necessary in clinical use.
\\However, since the Sputnik pump is a therapeutic option with the therapeutic goal of bridging to transplantation, it is conceivable that a patient may require prolonged support from the system. In such a case, it would be desirable not to restrict the patient's mobility and quality of life with an additional medical device. For this reason, a further development of the ILC approach, which would make the use of an ECG signal obsolete, would be desirable.
